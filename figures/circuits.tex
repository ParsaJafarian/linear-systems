\usepath[/home/adityam/Projects/presentations]

\enablemode[euler]
\environment env-fonts.tex
\usemodule[tikz,pgfplots]
\usemodule[circuitikz]

%\pgfplotsset{compat=1.11}
\ctikzset{bipoles/thickness=1}

\usetikzlibrary[shapes,arrows,arrows.meta,decorations.markings]
\usetikzlibrary[positioning,intersections]

\tikzstyle{block} = [draw, rectangle, 
    minimum height=3em, minimum width=6em, inner sep=2ex]
\tikzstyle{tightblock} = [draw, rectangle, 
    minimum height=3em, minimum width=3em, inner sep=2ex]
\tikzstyle{sum} = [draw, circle]
\tikzstyle{input} = []
\tikzstyle{output} = []


\tikzstyle{vecArrow} = [thick, decoration={markings,mark=at position
   1 with {\arrow[semithick]{open triangle 60}}},
   double distance=1.4pt, shorten >= 5.5pt,
   preaction = {decorate},
   postaction = {draw,line width=1.4pt, white,shorten >= 4.5pt}]

\tikzstyle{innerWhite} = [semithick, white,line width=1.4pt, shorten >= 4.5pt]


\starttext
\startTEXpage[offset=1dk]
\scale[scale=1500]{\startcircuitikz[american voltages,thick]
  \draw (0,0) to[R=$R$] (2,0) ;
\stopcircuitikz}
\stopTEXpage
\startTEXpage[offset=1dk]
\scale[scale=1500]{\startcircuitikz[american voltages,thick]
  \draw (0,0) to[C=$C$] (2,0) ;
\stopcircuitikz}
\stopTEXpage

\startTEXpage[offset=1dk]
\scale[scale=1500]{\startcircuitikz[american voltages,thick]
  \draw (0,0) to[L=$L$] (2,0) ;
\stopcircuitikz}
\stopTEXpage

\startTEXpage[offset=1dk]
\scale[scale=1500]{\startcircuitikz[american voltages,thick]
  \draw (0,0) to[V=$u(t)$,invert] (0,3) 
              to[R=$R$] (3,3) 
              to[L=$L$,-*] (6,3)
              node[label={[font=\tfxx]right:$v_C(t)$}] {}
              to[C=$C$] (6,0) -- (0,0);
  \draw[->,thick] plot [smooth, tension=1] coordinates { (1,2) (4.5,2) (4.5,1) (1,1) };

  \node (label) at (3,1.5) {$i_L(t)$};
\stopcircuitikz}
\stopTEXpage

\define\SMALL{\switchtobodyfont[10pt]}

\startTEXpage[offset=1dk]
\scale[scale=1500]{\startcircuitikz[american voltages,thick]
  \draw (0,0) to[V=$u(t)$,invert] (0,3) 
              to[L=$L$,i=\SMALL $i_L(t)$, -*] (3,3) 
              to[R=$R$,i=\SMALL $i_R(t)$] (3,0)
              to (0,0);
  \draw (3,3) to (5,3)
              node[label={[font=\SMALL]right:$v_C(t)$}] {}
              to[C=$C$,i=\SMALL $i_C(t)$,*-] (5,0)
              to (3,0);

\stopcircuitikz}
\stopTEXpage

\startTEXpage
\scale[scale=1500]{\starttikzpicture[thick]
      \node [input] (input) {$u(t)$};
      \node [tightblock, right=of input] (B) {$B$};
      \node [sum, right=of B] (sum1) {\tfc $+$};
      \node [tightblock, right=of sum1] (integrator) {$\dfrac{1}{s}$};
      \node [tightblock, below=8mm of integrator] (A) {$A$};
      \node [tightblock, right=16mm of integrator] (C) {$C$};
      \node [output, right=8mm of C] (output) {$y(t)$};

      \draw [->] (input) -- (B);
      \draw [vecArrow] (B) -- (sum1.west) node[left=5pt, above] {$+$};
      \draw [vecArrow] (sum1) -- (integrator);
      \draw [->] (C) -- (output);
      \draw [vecArrow] (A) -| (sum1.south) node[below=5pt, left] {$+$};
      \draw [vecArrow] (integrator) -- node[midway, coordinate, name=bend, label={[font=\SMALL]above:$x(t)$}] {} (C);
      \draw [vecArrow] (bend) |- (A);

      % 2nd pass
      \draw [innerWhite] (integrator) -- (C);
      \draw [innerWhite] (bend) |- (A);

    \stoptikzpicture}
\stopTEXpage

\startTEXpage[offset=1dk]
\scale[scale=1500]{\starttikzpicture[thick]
    \startaxis
        [xmin=-4, xmax=4,
         ymin=-4, ymax=4,
         grid=both,
         axis lines=middle,
         axis line style={latex-latex},
         ticklabel style={font=\SMALL,fill=white},
         minor tick num=1,
        ]

    \draw[-latex,very thick] (axis cs:0,0) -- (axis cs:2,3);
    \draw[-latex,red] (axis cs: 0,0) -- (axis cs:2, 0);
    \draw[-latex,red] (axis cs: 2,0) -- (axis cs:2, 3);
    \stopaxis
    \stoptikzpicture}
\stopTEXpage

\startTEXpage[offset=1dk]
\scale[scale=1500]{\starttikzpicture[thick]
    \startaxis
        [xmin=-4, xmax=4,
         ymin=-4, ymax=4,
         grid=both,
         axis x line=middle,
         axis y line=none,
         axis line style={latex-latex},
         ticklabel style={font=\SMALL,fill=white},
         minor tick num=1,
        ]


    \draw[latex-latex,thin,draw=darkgray] (axis cs:-4,-4) -- (axis cs: 4,4);
    \draw[thin,draw=black!20!white] (axis cs:-3,-4) -- (axis cs: 5,4);
    \draw[thin,draw=black!20!white] (axis cs:-2,-4) -- (axis cs: 6,4);
    \draw[thin,draw=black!20!white] (axis cs:-1,-4) -- (axis cs: 7,4);
    \draw[thin,draw=black!20!white] (axis cs:0,-4) -- (axis cs: 8,4);
    \draw[thin,draw=black!20!white] (axis cs:1,-4) -- (axis cs: 9,4);
    \draw[thin,draw=black!20!white] (axis cs:2,-4) -- (axis cs: 10,4);
    \draw[thin,draw=black!20!white] (axis cs:3,-4) -- (axis cs: 11,4);
    \draw[thin,draw=black!20!white] (axis cs:4,-4) -- (axis cs: 12,4);

    \draw[thin,draw=black!20!white] (axis cs:-5,-4) -- (axis cs: 3,4);
    \draw[thin,draw=black!20!white] (axis cs:-6,-4) -- (axis cs: 2,4);
    \draw[thin,draw=black!20!white] (axis cs:-7,-4) -- (axis cs: 1,4);
    \draw[thin,draw=black!20!white] (axis cs:-8,-4) -- (axis cs: 0,4);
    \draw[thin,draw=black!20!white] (axis cs:-9,-4) -- (axis cs: -1,4);
    \draw[thin,draw=black!20!white] (axis cs:-10,-4) -- (axis cs: -2,4);
    \draw[thin,draw=black!20!white] (axis cs:-11,-4) -- (axis cs: -3,4);
    \draw[thin,draw=black!20!white] (axis cs:-12,-4) -- (axis cs: -4,4);

    \draw[-latex,very thick] (axis cs:0,0) -- (axis cs:2,3);
    \draw[-latex,red] (axis cs: 0,0) -- (axis cs:3, 3);
    \draw[-latex,red] (axis cs: 3,3) -- (axis cs:2, 3);
    \stopaxis
    \stoptikzpicture}
\stopTEXpage

\startTEXpage[offset=1dk]
\switchtobodyfont[10pt]
\scale[scale=1500]{\startcircuitikz[american voltages,thick]
  \draw (0,0) to[V=$u(t)$,invert] (0,4) 
              to (2,4)
              to[R=$R_1$, -*] (2,2)
              node[label={left:$v_1(t)$}] {}
              to[C=$C_1$] (2,0)
              to (0,0);
  \draw (2,4) to (6,4)
              to[R=$R_2$, -*] (6,2)
              node[label={right:$v_2(t)$}] {}
              to[C=$C_2$] (6,0)
              to (2,0);

  \draw (2,2) to[R=$R_3$] (6,2);

\stopcircuitikz}
\stopTEXpage

\stoptext
